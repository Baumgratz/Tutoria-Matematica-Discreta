\documentclass[aspectratio=43]{beamer}
\usepackage[utf8]{inputenc}
\usepackage[T1]{fontenc}
\usepackage[brazil]{babel}
%\usepackage{tikz}
%\usepackage{graphics,amssymb,amsfonts,amsmath}
%\usepackage{amstext}
%\usepackage{amssymb}
%\usepackage{stmaryrd}
%\usepackage{fullpage}
\usepackage{amsmath,amsfonts,amssymb,amsthm}
\usepackage{proof}
\usepackage{color}

\usetheme{CambridgeUS}
% \usetheme{Pittsburgh}
\usecolortheme{beaver}
\usefonttheme[onlymath]{serif}
\beamertemplatenavigationsymbolsempty

% Colocando numero de paginas no slide
%\setbeamertemplate{footline}[frame number]

\newcommand{\ria}{$\rightarrow$}
\newcommand{\fall}{$\forall$}
\newcommand{\ex}{$\exists$}
\newcommand{\nao}{$\neg$}
\newcommand{\nex}{\nao\ex}
\newcommand{\nfall}{\nao\fall}
\newcommand{\andd}{$\wedge$}
\newcommand{\orr}{$\vee$}
\newcommand{\gam}{$\Gamma$}
\newcommand{\nat}{$\mathbb{N}$}
\newcommand{\Nat}{\mathbb{N}}
\newcommand{\inteiro}{$\mathbb{Z}$}
\newcommand{\real}{$\mathbb{R}$}

\title[\sc{Resolu\c c\~ao}]{Resolu\c c\~ao 3.7-1}
\author[Guilherme Baumgratz Figueiroa]{Guilherme Baumgratz Figueiroa}
\institute[UFOP]{Universidade Federal de Ouro Preto} % opcional
\date{}

\begin{document}
	
	\begin{frame}
		\titlepage
	\end{frame}

	\section{Exercic\'io}

	\begin{frame}[fragile]
    \begin{enumerate}[1.]

		\item Para cada uma das fórmulas a seguir, indique se ela é verdadeira ou falsa, quando o universo de discurso é cada um dos seguintes: \nat : conjunto dos n\'umeros naturais, \inteiro : conjunto dos n\'umeros inteiros e \real\ conjunto dos n\'umeros reais. Considere que os símbolos matemáticos possuem o significado usual.\\
        \centering
        \vspace{20pt}
        \begin{tabular}{c|l|c|c|c|}

        	\ & F\'ormula & \nat & \inteiro & \real \\\hline
        	a & $\exists x.x^{2} = 2$ & \ & \ & \  \\
        	b & $\forall x. \exists y. x^{2} = y$ & \ & \ & \  \\
        	c & $\forall x. x \neq 0 \to \exists y.xy = 1 $ & \ & \ & \  \\
        	d & $\exists x. \exists y.(x + 2y^{2} = 2) \land (2x + 4y = 5)$ & \ & \ & \  \\
		\end{tabular}
        
	\end{enumerate}
		
	\end{frame}
    
    \subsection{DICA!!}
	
   	\begin{frame}[fragile]
    
    \centering
    \huge \nat\ $\subset$ \inteiro\ $\subset$ \real

	\end{frame}


    \subsection{$\exists x.x^{2} = 2$}
    
    \begin{frame}[fragile]
    
    \begin{enumerate}[a)]
		\item $\exists x.x^{2} = 2$
	\end{enumerate}
    \pause
    \nat \ e \inteiro \\
    ${x = 0 \Rightarrow x^2 = 0}$ \\
    \pause
    ${x = 1 \Rightarrow x^2 = 1}$ \\
    \pause
    ${x = 2 \Rightarrow x^2 = 4}$ \\
    $\vdots$ \\
	\pause
    \vspace{5pt}
    \real \\
    ${x=\sqrt[2]{2} \Rightarrow x^2 = (\sqrt[2]{2})^2 \Rightarrow x^2 = 2}$ \\
    \pause
    Ent\~ao, $\exists x.x^{2} = 2$, pertence aos \real, por\'em n\~ao pertence aos \inteiro\ nem aos \nat .\\
    
	\end{frame}
    
    \subsection{$\forall x.\exists y. x^{2} = y$}
    
    \begin{frame}[fragile]
    
    \begin{enumerate}[b)]
		\item $\forall x.\exists y. x^{2} = y$
	\end{enumerate}
    \pause
    Nos n\'umeros naturais, ou o n\'umero \'e par, tal que 
    \begin{center}
    $\forall n. n \in \Nat \to \exists p.p \in \Nat .\ 2n = p$,
	\end{center}
    ou o n\'umero \' impar, tal que 
    \begin{center}    
    $\forall n. n \in \Nat \to \exists i.i \in \Nat .\ 2n +1 = i$,
    \end{center}
    ou \'e o n\'umero zero.\\
    Sabendo disso, vamos para os casos:\\
	
	\end{frame}
    
    \begin{frame}[fragile]
     \begin{enumerate}[b)]
		\item $\forall x.\exists y. x^{2} = y$
	\end{enumerate}
    
    Caso base ($x=0$): \\
    \vspace{5pt}
	${x = 0 \pause \Rightarrow x^2 = 0 \pause \Rightarrow 0 \in \Nat}$\\
    \vspace{5pt}
    \pause
    Caso x par($x=2n$):\\
    \vspace{5pt}
    ${x = 2n \pause \Rightarrow x^2 = (2n)^2 \pause \Rightarrow x^2 = 4n^2 \pause \Rightarrow x^2 = 2(2n^2) \pause \rightarrow y = 2n^2 \pause \rightarrow x^2 = 2y}$\\
    \vspace{5pt}
    \pause
    Caso x \'impar($x=2n+1$):\\
    \vspace{5pt}
    ${x = 2n+1 \pause\Rightarrow x^2 = (2n+1)^2 \pause \Rightarrow x^2 = 4n^2 + 4n + 1 \pause \Rightarrow}$\\
    ${x^2 = 2(2n^2) + 2(2n) + 1 \pause \rightarrow y = 2n^2 \land z = 2n \pause\rightarrow}$\\
    ${x^2 = (2y) + (2z +1)}$ \\
    \vspace{5pt}
    \pause
    Portando, $\forall x.\exists y. x^{2} = y \ \in \Nat$ o que, consequentemente, faz a expressão pertencer ao \inteiro \ e ao \real  .

	\end{frame}
    
    \subsection{$\forall x. x \neq 0 \to \exists y.xy = 1 $}
    
    \begin{frame}[fragile]
    \begin{enumerate}[c)]
		\item $\forall x. x \neq 0 \to \exists y.xy = 1 $
	\end{enumerate}
    \vspace{5pt}
    \pause
    Para que essa senten\c ca seja verdadeira, temos que fazer o $y$ seja o complemento de $x$,
    ou seja $y = \frac{1}{x}$.\\
    Sendo  $y = \frac{1}{x}$, então a sentença é verdadeira somente para o \real .
	
	\end{frame}
    
    \subsection{$\exists x. \exists y.(x + 2y^{2} = 2) \land (2x + 4y = 5)$}
    
    \begin{frame}
	\begin{enumerate}[d)]
		\item $\exists x. \exists y.(x + 2y^{2} = 2) \land (2x + 4y = 5)$
	\end{enumerate}
    \pause
    \nat \ e \inteiro \\
   	No \nat \ e no \inteiro, ele n\~ao consegui satisfazer $2x + 4y =5$.\\
    Vejamos a tabela de valores:\\
    \begin{center}

    \begin{tabular}{c|c|c|c}
		 \nat & \inteiro & $x$ & $y$ \\ 
        \hline$-$ & x & $-3$ & \pause $\frac{11}{4}$\\
        \hline$-$ & x & $-2$ & \pause $\frac{9}{4}$\\
        \hline$-$ & x & $-1$ & \pause $\frac{7}{4}$\\
        \hline x & x & $0$ & \pause $\frac{5}{4}$\\
        \hline x & x & $1$ & \pause $\frac{3}{4}$\\
        \hline x & x & $2$ & \pause $\frac{1}{4}$\\
        \hline x & x & $3$ & \pause $-\frac{1}{4}$\\
        
	\end{tabular}
	\end{center}
      
	\end{frame}
    
    \begin{frame}
	\begin{enumerate}[d)]
		\item $\exists x. \exists y.(x + 2y^{2} = 2) \land (2x + 4y = 5)$
	\end{enumerate}
    \real \\
    Todos os valores anteriores testados estam no \real, por\'em nenhum deles funciona na outra form\'ula. Existe um \'unico valor que se aplica a forma, e ela foi descoberta usando sistema de duas equa\c c\~oes.\\
    Os valores s\~ao:\\
    \begin{center}

    $x = \frac{3}{2}$\\
    \vspace{10pt}
    $y = \frac{1}{2}$
    \end{center}
	Ou seja, a express\~ao $\exists x. \exists y.(x + 2y^{2} = 2) \land (2x + 4y = 5)$ est\'a contido no \real.
	\end{frame}
    
\end{document}
