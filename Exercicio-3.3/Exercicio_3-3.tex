\documentclass[aspectratio=43]{beamer}
\usepackage[utf8]{inputenc}
\usepackage[T1]{fontenc}
\usepackage[brazil]{babel}
\usepackage{tikz}

\usetheme{CambridgeUS}
% \usetheme{Pittsburgh}
\usecolortheme{beaver}
\usefonttheme[onlymath]{serif}
\beamertemplatenavigationsymbolsempty

% Colocando numero de paginas no slide
%\setbeamertemplate{footline}[frame number]

\newcommand{\ria}{$\rightarrow$}
\newcommand{\fall}{$\forall$}
\newcommand{\ex}{$\exists$}
\newcommand{\nao}{$\neg$}
\newcommand{\nex}{\nao\ex}
\newcommand{\nfall}{\nao\fall}
%\newcommand{\and}{\wedge}
%\newcommand{\or}{\vee}

\title[\sc{Resolu\c c\~ao}]{Resolu\c c\~ao 3.3-3}
\author[Guilherme Baumgratz Figueiroa]{Guilherme Baumgratz Figueiroa}
\institute[UFOP]{Universidade Federal de Ouro Preto} % opcional
\date{}

\begin{document}
	
\begin{frame}
	\titlepage
\end{frame}

\section{Exercic\'io}

\begin{frame}%\frametitle{Exercic\'io}
	3.Considerando como universo de discurso o conjunto de alunos e professores de uma universidade e os seguintes predicados:
	\begin{table}[h]
		\begin{tabular}{|c|l|}
			$A(x, y)$ & $x$ admira $y$ \\
			$S(x, y)$ & $x$ estava presente em $y$ \\
			$P(x)$ & $x$ \'e professor \\ 
			$E(x)$ & $x$ \'e estudante \\
			$L(x)$ & $x$ \'e aula \\
		\end{tabular}
	\end{table}
	e a constante $Maria$, representa as seguintes sentenças como fórmulas da lógica de predicados.
	
\end{frame}

\begin{frame}
	\begin{enumerate}[a)]
		
		\item Maria admira todo professor \\
		\pause
		Resposta: $\forall x.P(x) \rightarrow A(Maria, x)$\\
		\pause
		
		\item Algum professor admira Maria \\
		\pause
		Resposta: $\exists x.P(x) \rightarrow A(x, Maria)$ \\
		\pause
		
		\item Maria admira a si mesma \\
		\pause
		Resposta: $ A(Maria, Maria)$ \\
				
	\end{enumerate}
		
\end{frame}

\begin{frame}
	\begin{enumerate}[d)]
		\item Nenhum estudante estava presente em todas as aulas \\
		\pause
		Resposta: $\neg\exists x.E(x).\forall y.L(y).S(x, y)$ \\
		\pause
		
		\item Nenhuma aula teve a presen\c ca de todos os estudantes \\
		\pause
		Resposta:  $\neg\exists x.L(x).\forall y.E(y).S(y, x)$
		\pause
		
		\item Nenhuma aula teve a presen\c ca de qualquer estudantes \\
		\pause
		Resposta:  $\neg\exists x.L(x).\neg\exists y.E(y).S(y, x)$
	\end{enumerate}
\end{frame}
	
\end{document}
